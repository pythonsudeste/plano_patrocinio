\documentclass[12pt]{article}
\usepackage[brazil]{babel}
\usepackage[utf8]{inputenc}
\usepackage[T1]{fontenc}
\usepackage{graphicx}
\usepackage{xcolor}
\usepackage{enumitem}

\include{defs}

\usepackage{lipsum}

%%%%%%%%%%%%%%%
% Title Page
\title{Rio de Janeiro , 2017}
\author{pythonsudeste@gmail.com \\ Bernardo Fontes (021 99629 1621)}
\date{\today}
%%%%%%%%%%%%%%%

\begin{document}
\maketitle


\section{Python Sudeste 2017}

Python Sudeste é um evento voltado para o público adepto da linguagem de programação Python que tem como objetivo reunir desenvolvedores de Minas Gerais, Rio de Janeiro, São Paulo e Espírito Santo favorecendo a troca de experiências e conhecimentos.

Com a Python Sudeste 2017, visamos difundir a linguagem de programação Python para iniciantes e desenvolvedores que utilizem outras linguagens de programação, trazendo novas pessoas para a comunidade, além de promover a troca de conhecimento e interação entre os desenvolvedores.

A conferência é um evento sem fins lucrativos, organizada pela comunidade e para a comunidade, o que garante a possibilidade de um ingresso muito mais barato em comparação a outros eventos de tecnologia e portanto acessível a um maior número de pessoas.

Python é uma linguagem amplamente utilizada em diversos segmentos econômicos, resolvendo problemas do dia a dia de forma robusta, eficiente e de fácil acesso. É uma das dez linguagens mais utilizadas no mundo segundo o índice TIOBE, que classifica as linguagens de programação pela frequência de pesquisa na Web.

\section{Porque patrocinar a Python Sudeste?}

Os patrocinadores são responsáveis por fazer a Python Sudeste possível com baixos preços de ingressos. É necessária ajuda financeira paras diversas atividades que ocorrem durante o evento e infraestrutura. As empresas que apoiam a Python Sudeste, por sua vez, apoiam toda a comunidade Python tornando possível para muitos desenvolvedores participarem do evento, apresentarem conteúdos relevantes e também nos permitem disponibilizar material online para os demais membros da comunidade aproveitarem as palestras de casa.

E essa contribuição acompanha um excelente conjunto de benefícios aos nossos patrocinadores, como:
\begin{description}[align=right,labelwidth=4cm]
    \item [Visibilidade:] Atingir diretamente pessoas com um interesse real no tema da conferência e executar marketing direcionado.
    \item [Oportunidades:] Uma conferência mostra-se um ambiente muito proveitoso para atrair novos projetos digitais.
    \item [Produtividade:] Eventos de software servem como reciclagem e treinamento continuo da equipe e aumenta a performance dos desenvolvedores que por sua vez estão sempre em contato com conhecimentos novos sobre boas práticas, testes, projetos open source...
    \item [Contratação:] Tire partido da conferência de recrutamento, afinal que local é melhor pra encontrar bons profissionais que em um evento de software?
    \item [Patrocínio:] Apoie uma grande comunidade de código aberto!
\end{description}

\section{Alcance em Mídias Sociais: Facebook}
\begin{itemize}[label={}]
    \item \emph{PythonSudeste} (oficial do evento) : 143 seguidores
    \item \emph{PythonRio} : 1561 seguidores
    \item \emph{UAI Python - Python MG } : 768 seguidores
    \item \emph{Grupy-SP} :  614 seguidores
    \item \emph{Grupy-RP} : 190 seguidores
\end{itemize}

\section{Atuação da comunidade Python na Região Sudeste}

Os desenvolvedores da região sudeste brasileira contribuíram de maneira substancial para o crescimento da comunidade Python no Brasil e na sua expressiva atuação.

Hoje, temos mais de 100 projetos open source mantidos por pythonistas dos 4 estados do Sudeste, centenas de empresas utilizando Python com diferentes objetivos como: Análise de dados, Desenvolvimento de aplicações Web, Internet das Coisas, etc.

\section{A cidade: Rio de Janeiro}

A cidade do Rio de Janeiro concentra um grande número de desenvolvedores e empresas que utilizam Python. A comunidade de desenvolvedores Python vem realizando eventos na cidade desde 2007 e hoje organiza-se um encontro mensal há mais de um ano com público médio de 100 pessoas, sempre com novos participantes. A PythonRio possui grupos de discussão com mais de 300 desenvolvedores interagindo diariamente sobre questões da linguagem.


\section{Inclusão e diversidade}

Um dos objetivos do evento é garantir a inclusão de novas pessoas independente de gênero, orientação sexual, etnia ou credo. Para isso, buscamos assegurar um ambiente seguro e livre de preconceitos que possibilite o máximo de interação e abordagens dos mais variados temas dentro do universo Python.

Temos concomitantemente por objetivo promover uma maior participação no evento através de ajudas de custo para aqueles que não podem arcar com o ingresso ou outras eventuais despesas, possibilitando uma maior integração de pessoas não inseridas no mercado de trabalho de tecnologia.

\section{O evento}

O evento ocorrerá nos dias 05 e 06 de maio na Universidade Veiga de Almeida - Campus Maracanã.

Endereço: R. Ibituruna, 108 - Maracanã, Rio de Janeiro - RJ, 20271-020

Estrutura: Auditório e sala de interação.

\begin{itemize}[label={}]
    \item Projetores, multimídia, microfones e caixas de som.
    \item Filmagem e fotos do evento.
\end{itemize}

Público esperado: 250 participantes

\section{Cotas de Patrocínio}
\subsection{Patrocinador Diamante}
  \begin{itemize}[label={}]
  \setlength\itemsep{0.0em}
        \item Logo no site do evento
        \item Logo nos intervalos
        \item Logo nos banners
        \item Logo nos crachás
        \item Post do patrocinador nas redes sociais$^2$
        \item Divulgação da marca no evento $^3$
        \item Cinco inscrições de participantes no evento.
        \item Valor: $R\$ 4.000,00$
  \end{itemize}
\subsection{Patrocinador Platina}
  \begin{itemize}[label={}]
  \setlength\itemsep{0.0em}
        \item Logo no site do evento
        \item Logo nos intervalos
        \item Logo nos banners
        \item Logo nos crachás
        \item Divulgação da marca no evento$^3$
        \item Quatro inscrições de participantes no evento.
        \item Valor: $R\$ 3.000,00$
  \end{itemize}
\subsection{Patrocinador Ouro}
  \begin{itemize}[label={}]
  \setlength\itemsep{0.0em}
        \item Logo no site do evento
        \item Logo nos intervalos
        \item Logo nos banners
        \item Divulgação da marca no evento$^3$
        \item Três inscrições de participantes no evento.
        \item Valor: $R\$ 2.000,00$
  \end{itemize}
\subsection{Patrocinador Prata}
    \begin{itemize}[label={}]
    \setlength\itemsep{0.0em}
        \item Logo no site do evento
        \item Logo nos intervalos
        \item Logo nos banners
        \item Duas inscrições de participantes no evento.
        \item Valor $R\$ 1.000,00$
    \end{itemize}
\subsection{Patrocinador bronze}
    \begin{itemize}[label={}]
    \setlength\itemsep{0.0em}
        \item Logo no site do evento
        \item Logo nos intervalos
        \item Uma inscrição de participante no evento.
        \item Valor $R\$ 300,00$
    \end{itemize}

$^1$ Os benefícios de logo no site do evento, logo nos intervalos, logo nos banners e logo nos crachás serão ordenados e com tamanhos proporcionais a cota de patrocínio. Desta forma, os patrocinadores da cota Diamante serão exibidos primeiro e com maior destaque e a gradação de tamanho irá diminuir por categoria para cada um dos benefícios citados.

$^2$ O post do patrocinador será feito pela página do facebook e do twitter nas contas na Python Sudeste e Python Rio. Na página do facebook será feito um post promovido aumentando o alcance.

$^3$ Serão permitidas as atividades de: Panfletagem, distribuição de brindes e a exposição de um  banner com dimensões de no máximo 1 metro de largura por 2 metros de  altura.

\end{document}
